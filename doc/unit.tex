\documentclass{article}

% Input packages & formatting
% Packages

% Math packages
\usepackage{amsmath} % Extended math functions
\usepackage{amssymb} % Extended math symbols (loads in amsfonts)
\usepackage{bm} % Bold math symbols
\usepackage{mathtools}

% Figure packages
\usepackage{caption} % Caption formatting for university standard
\usepackage{graphicx} % includegraphics command
\usepackage{subcaption} % Subfigures
\usepackage[section]{placeins} % Place floats in section
\usepackage{wrapfig}

% Table packages
\usepackage{booktabs} % Better tables
\usepackage{bigstrut} % Merged table cells
\usepackage{longtable} % Tables which overflow into next page
\usepackage{array}
\usepackage{colortbl} % Color table cells
\usepackage{makecell}
\usepackage{multirow}

% Fonts
\usepackage{lmodern} % Use latin modern rather than computer modern. Better for font encoding.
\usepackage[T1]{fontenc} % Allow text to be searchable in output

% Other packages
\usepackage{appendix} % Appendix environment
\usepackage{nextpage} % Cleartooddpage command
%\usepackage[square,comma,sort,numbers]{natbib} % Reference formatting
\usepackage{setspace} % Line spacing
\usepackage{listings} % Display code with syntax highlighting
\usepackage{upquote} % Vertical quotes in verbatim
\usepackage{xcolor} % Colors
\usepackage{titlesec} % Header spacing
\usepackage{xparse} % for tcolorbox
\usepackage[listings]{tcolorbox} % Colored boxes for highlighting syntax
\tcbuselibrary{breakable}
\tcbuselibrary{skins}
\usepackage{enumitem} % better enumerate/itemize options
\usepackage{fancyhdr}
\usepackage{multicol}
\usepackage{ifthen}
\usepackage{xstring}

% Table of contents
\usepackage{imakeidx} % Index page
\usepackage{tocloft} % Control of table of contents
\usepackage[nottoc]{tocbibind} % Adds bibliography, table of tables, table of figures, to table of contents
\usepackage[bookmarks,linktocpage,hidelinks]{hyperref} % Hyperlinks for sections, figures, etc.

% Formatting
% Page format
\setlength{\oddsidemargin}{0.00in}  % Left side margin for odd numbered pages
\setlength{\evensidemargin}{0.00in} % Right side margin for even numbered pages
\setlength{\topmargin}{0.00in}      % Top margin
\setlength{\headheight}{0.20in}     % Header height
\setlength{\headsep}{0.20in}        % Separation between header and main text
\setlength{\topskip}{0.00in}        % Top skip
\setlength{\textwidth}{6.50in}      % Width of the text
\setlength{\textheight}{8.50in}     % Height of the text
\setlength{\footskip}{0.50in}       % Foot skip
\setlength{\parindent}{0.00in}      % First line indentation
\setlength{\parskip}{6pt}        % Space between two paragraphs

% Captions (figures, tables, etc.)
\setlength{\floatsep}{\parskip}          % Space left between floats.
\setlength{\textfloatsep}{\floatsep}   % Space between last top float
% or first bottom float and the text
\setlength{\intextsep}{\floatsep}      % Space left on top and bottom
% of an in-text float
\setlength{\abovecaptionskip}{0.1in plus 0.25in}  % Space above caption
\setlength{\belowcaptionskip}{0.1in plus 0.25in}  % Space below caption
\setlength{\captionmargin}{0.50in}     % Left/Right margin for caption
\setlength{\abovedisplayskip}{0.00in plus 0.25in} % Space before Math stuff
\setlength{\belowdisplayskip}{0.00in plus 0.25in} % Space after Math stuff
\setlength{\arraycolsep}{0.10in}       % Gap between columns of an array
\setlength{\jot}{0.10in}                % Gap between multiline equations
\setlength{\itemsep}{0.10in}           % Space between successive items

% Counters (no section numbering)
\setcounter{tocdepth}{3}
\setcounter{secnumdepth}{0}

% Spacing
\setstretch{1.5}

\titlespacing*{\section}{0cm}{6pt}{6pt}[0cm]
\titlespacing*{\subsection}{0cm}{6pt}{6pt}[0cm]
\titlespacing*{\subsubsection}{0cm}{6pt}{6pt}[0cm]

\titleformat{\section}
{\sffamily\huge}{}{0pt}{\titlerule\vspace{-0.2cm}}
\titleformat{\subsection}
{\sffamily\itshape\Large}{}{0pt}{}

% Macro for syntax
\newtcolorbox{syntax}{
    size=small,
    sharp corners,
    colframe=black,
    colback=yellow,
    fontupper=\bfseries\ttfamily
}

% Macro for argument table
\newenvironment{args}{
    \begin{tabular}{>{\bfseries\ttfamily}p{0.25\linewidth} p{0.69\linewidth}}
    }{
    \end{tabular}\par
    \vspace{0.5\baselineskip}
}

% Note: Requires packages "listing", "xcolor", and "textcomp"
\lstdefinelanguage{verbatim}{
    basicstyle=\ttfamily\small,
    xleftmargin=9pt,
    xrightmargin=9pt,
    columns=fullflexible,
    keepspaces=true,
    breaklines=true
}

% Example code
\AtBeginDocument{
\newtcolorbox[blend into=listings]{example}[2][]{
    colback=blue!3!white,
    colframe=black,
    colbacktitle=blue!15!white,
    coltitle=black,
    sharp corners,
    enhanced,
    breakable,
    size=small,
    before upper={
        \setstretch{1.0}\lstset{language=verbatim}\vspace{3pt}\textsf{\textit{Code:}}
    },
    subtitle style={
        colback=blue!20!white,
        fonttitle=\sffamily
    },
    before lower={
        \setstretch{1.0}\lstset{language=verbatim}\vspace{3pt}\textsf{\textit{Output:}}
    },
    fonttitle=\sffamily,
    title={#2},
    #1
}
}

% Links to sub and subsub commands - optional boolean argument, default true. if false, only displays subcmd.

% Commands (and command ensembles)
\newcommand{\command}[1]{\protect\hypertarget{#1}{#1}\index{#1}}
\newcommand{\subcommand}[2]{\protect\hypertarget{#1 #2}{#1 #2}\index{#1!#2}}
\newcommand{\cmdlink}[1]{\protect\hyperlink{#1}{\textit{#1}}}
\newcommand{\subcmdlink}[3][1]{\protect\hyperlink{#2 #3}{\ifnum#1=1\relax\textit{#2 #3}\else\textit{#3}\fi}}

% Methods (first arg is class)
\newcommand{\method}[2]{\protect\hypertarget{$#1Obj #2}{\$#1Obj #2}\index{#1 methods!#2}}
\newcommand{\methodlink}[3][1]{\protect\hyperlink{$#2Obj #3}{\ifnum#1=1\relax\textit{\$#2Obj #3}\else\textit{#3}\fi}}

% Macros for figure/table names
\newcommand{\fig}{\figurename\ }
\newcommand{\figs}{\figurename s }
\newcommand{\tbl}{\tablename\ }
\newcommand{\tbls}{\tablename s }
\newcommand{\eq}{Eq. }
\newcommand{\eqs}{Eqs. }
\renewcommand{\lstlistingname}{Example}% Listing -> Example
\renewcommand{\lstlistlistingname}{List of \lstlistingname s}% List of Listings -> List of Examples
\newcommand{\ex}{Example }
\newcommand{\exs}{Examples }
\newcommand{\var}[1]{\texttt{\textbf{\$#1}}}

% Header/footer
\renewcommand{\headrulewidth}{0pt}

% Changes to hyperlinks (URLs)
\renewcommand\UrlFont{\color{blue}\rmfamily}

% New column type 
% https://tex.stackexchange.com/questions/75717/how-can-i-mix-itemize-and-tabular-environments
\newcolumntype{L}{>{\labelitemi~~}l<{}}
\newcommand{\version}{0.1.2}

\renewcommand{\cleartooddpage}[1][]{\ignorespaces} % single side
\newcommand{\caret}{$^\wedge$}

% Other macros

\title{\Huge Unit System (unit) \\\small Version \version}
\author{Alex Baker\\\small\url{https://github.com/ambaker1/unit}}
\date{\small\today}
\begin{document}
\maketitle
\begin{abstract}
The ``unit'' package provides a framework for defining consistent units for engineering analysis with base units of force, length, time and thermodynamic temperature. 
Units in the ``unit'' package are defined with both a conversion value and a dimension vector - which allows for definition of any base unit system and unit conversion that checks for consistent units. 
In addition to conversion utilities, conversion values to user-defined base units can be imported into an analysis as variables for use in Tcl expressions. 
For convenience, the ``unit'' package provides a built-in set of typically used units for engineering analysis, but also provides tools to add new units, combine units, and remove units. 
The ``unit'' package also introduces a new notation for expressing combinations of units, which allows for easy introspection and manipulation of the unit system. 
\end{abstract}
\clearpage
\section{General}
The unit package supplies one command, \cmdlink{unit}, which has a variety of subcommands for unit manipulation.
\begin{syntax}
\command{unit} \$subcommand \$arg1 \$arg2 ...
\end{syntax}
\begin{args}
\$subcommand & Subcommand for main command. Options listed below: \\
\quad combine & \quad Combine units using unit string notation. \\
\quad convert & \quad Convert between unit strings (with dimensionality check). \\
\quad exists & \quad Check if a unit is defined. \\
\quad expr & \quad Get unit conversion value associated with unit string. \\
\quad import & \quad Import units as variables. \\
\quad info & \quad Get dictionary of conversion value and dimension for unit. \\
\quad names & \quad Get list of defined units, optionally with dimensions specified. \\
\quad new & \quad Create a unit, with specified dimensions. \\
\quad remove & \quad Remove units. \\
\quad string & \quad Get unit string for specified dimensions. \\
\quad system & \quad Define or redefine the base units for the system of units. \\
\quad type & \quad Define, remove, or query unit type dimensions. \\
\quad types & \quad Get unit types, optionally with dimensions specified. \\
\quad wipe & \quad Clear all units and forget unit system. \\
\$arg1 \$arg2 ... & Arguments for subcommand.
\end{args}
\clearpage
\section{Defining the System of Units}
The command \subcmdlink{unit}{system} queries, defines, or redefines the base units (force, length, time and temperature). 
Note that force is not technically a base unit, but it is convenient to consider it as one for engineering analysis. 
If called with no input arguments, it simply queries the current base units. 
Otherwise, it defines the base units.
If initializing a unit system (after calling \subcmdlink{unit}{wipe}), the specified base units will be created and initialized with values of 1.0. 
If redefining the unit system, the specified base units must already be defined in the existing unit system.
\begin{syntax}
\subcommand{unit}{system} <default> <\$forceUnit \$lengthUnit \$timeUnit <\$tempUnit>{}>
\end{syntax}
\begin{args}
default & Option to wipe the unit system and restore default built-in unit system (N m sec K). \\
\$forceUnit & Base unit for force (built-in options: in ft mm cm m) \\
\$lengthUnit & Base unit for length (built-in options: kip lbf kN N kgf tonf) \\
\$timeUnit & Base unit for time (built-in options: msec sec min hr) \\
\$tempUnit & Base unit for temperature (built-in options: K R) (default K)
\end{args}
\begin{example}{Defining base unit system}
\begin{lstlisting}
unit system kip in sec
puts [unit system]
\end{lstlisting}
\tcblower
\begin{lstlisting}
kip in sec K
\end{lstlisting}
\end{example}
Note for OpenSees users: If all quantities in your model are defined with units, the base units can be changed and all output quantities will be in the new base units. 
This is especially useful if you need to have results in US and SI units. A note of warning, however, is that units can show up in unexpected places.
For example, convergence tests are essentially in base units. \textit{NormUnbalance} \& \textit{RelativeNormUnbalance} test tolerances are in force units, \textit{NormDispIncr}, \textit{RelativeNormDispIncr} \& \textit{RelativeTotalNormDispIncr} test tolerances are in length units, and \textit{EnergyIncr} \& \textit{RelativeEnergyIncr} test tolerances are in energy (force*length) units. 
Care must be taken to ensure that model is completely defined with units if it is desired to be able to change default base units.

\clearpage
\section{Unit Dimensions and Types}
The unit package stores both a conversion value and a unit dimension vector for each unit and combination of units.
The dimension vector contains four values, representing the exponent on force, length, time, and temperature, respectively. 
For example, a force unit would have the dimension vector ``\texttt{\{1 0 0 0\}}'', and an acceleration unit would have the dimension vector ``\texttt{\{0 1 -2 0\}}''.
Many of the commands in this module require dimension input, so to make input easier, unit types can be used instead.
A unit type is a string that represents a dimension vector, such as using ``\texttt{force}'' as an alias for ``\texttt{\{1 0 0 0\}}''.
Additionally, rather than a single dimension vector or dimension type, dimension vectors and types can be combined with exponents, in a key-value pair fashion.
For example, the inputs  ``\texttt{acceleration}'', ``\texttt{\{0 1 -2 0\}}'', ``\texttt{length 1 time -2}'', and ``\texttt{\{0 1 0 0\} 1 \{0 0 1 0\} -2}'' are all equivalent.

\subsection{Create, Query, and Remove Unit Types}
The command \subcmdlink{unit}{type} can be used to add new unit types, remove existing unit types, and query the dimension vectors associated with the unit type. 
\begin{syntax}
\subcommand{unit}{type} \$typeName <\$dimArg1 \$dimArg2 ...>
\end{syntax}
\begin{args}
\$typeName & Unit type to query or modify \\
\$dimArg1 \$dimArg2 ... & Unit dimension arguments. 
If empty, it simply queries the dimension of the unit type. 
If the input is the blank string ``'', the unit type will be removed. 
Otherwise, it will define (or redefine) the unit type.
\end{args}

\begin{example}{Defining a new unit type}
\begin{lstlisting}
unit type momentum mass 1 velocity 1
puts [unit type momentum]
\end{lstlisting}
\tcblower
\begin{lstlisting}
1 0 1 0
\end{lstlisting}
\end{example}
Note: the base unit types (force, length, time, temp) and the ``constant'' unit type cannot be modified or removed. 
\clearpage
\subsection{Get Unit Types}
The command \subcmdlink{unit}{types} returns a list of all unit types, with the option to specify the dimension.
\begin{syntax}
\subcommand{unit}{types} <\$dimArg1 \$dimArg2 ...>
\end{syntax}
\begin{args}
\$dimArg1 \$dimArg2 ... & Unit dimension arguments. By default, returns all unit types. If valid dimension arguments are provided, returns all unit types matching the provided dimensions.
\end{args}
\begin{example}{Query unit types}
\begin{lstlisting}
puts [unit types mass 1 accel 1]
puts [unit types mass 1 length 2]
\end{lstlisting}
\tcblower
\begin{lstlisting}
force
rotMass inertia
\end{lstlisting}
\end{example}
\clearpage
\section{Built-In Units}
Upon loading in the unit module, the unit system will already be defined in SI units (N m sec K) and a built-in set of units will be provided. 
If a different base unit system is desired, the unit system can be redefined with the command \subcmdlink{unit}{system}, and all the built-in units will be redefined in the new unit system. 
Most of the built-in units, with their conversion values to SI units, are shown below:\par
\begin{center}
\begin{tabular}{ccc}
\begin{tabular}[t]{ll}
\hline
\multicolumn{2}{c}{Force Units} \\
\hline
N & 1 \\
kN & 1000 \\
MN & 1000000 \\
kgf & 9.80665 \\
tonf & 9806.65 \\
lbf & 4.4482216152605 \\
kip & 4448.2216152605 \\
\hline
\multicolumn{2}{c}{Length Units} \\
\hline
m & 1 \\
mm & 0.001 \\
cm & 0.01 \\
km & 1000 \\
in & 0.0254 \\
ft & 0.3048 \\
yd & 0.9144
\end{tabular}
& \quad
\begin{tabular}[t]{ll}
\hline
\multicolumn{2}{c}{Time Units} \\
\hline
sec & 1 \\
msec & 0.001 \\
min & 60 \\
hr & 3600 \\
\hline
\multicolumn{2}{c}{Temperature Units} \\
\hline
K & 1 \\
R & 0.555555555555556 \\
\hline
\multicolumn{2}{c}{Standard Gravity} \\
\hline
g & 9.80665 \\
\hline
\multicolumn{2}{c}{Mass Units} \\
\hline
kg & 1 \\
gram & 0.001 \\
Mg & 1000 \\
slug & 14.5939029372064 \\
lbm & 0.45359237
\end{tabular}
& \quad
\begin{tabular}[t]{ll}
\hline
\multicolumn{2}{c}{Stress/Pressure Units} \\
\hline
ksi & 6894757.29316836 \\
ksf & 47880.2589803358 \\
psi & 6894.75729316836 \\
psf & 47.8802589803358 \\
Pa & 1 \\
kPa & 1000 \\
MPa & 1000000 \\
GPa & 1000000000 \\
atm & 101325 \\
\hline
\multicolumn{2}{c}{Constants} \\
\hline
pi & 3.14159265358979 \\
e & 2.71828182845905 \\
rad & 1 \\
deg & 0.0174532925199433 \\
cycle & 6.28318530717959 
\end{tabular}
\end{tabular}
\end{center}
Other unit types also have built-in units, such as Hz for frequency and kJ for energy.
For brevity, these additional units and their conversion values are omitted, but all built-in units and their conversion values can be easily queried within the program.

Note: If \subcmdlink{unit}{wipe} is called, these built-in units will be removed, and the unit system will be blank. They can be restored by calling \textit{unit system default}.
\clearpage
\section{Unit System Information}
Besides querying the base units with \subcmdlink{unit}{system}, the unit module provides commands for getting a list of other defined units, checking existence of units, and querying unit conversion values and unit base unit dimensions.
\subsection{Get Listing of All Units}
The command \subcmdlink{unit}{names} returns a list of all defined units, with the option to specify dimension type.
\begin{syntax}
\subcommand{unit}{names} <\$dimArg1 \$dimArg2 ...>
\end{syntax}
\begin{args}
\$dimArg1 \$dimArg2 ... & Unit dimension arguments. By default, returns all unit names. If dimension arguments are provided, returns all matching units.
\end{args}
\subsection{Checking if a Unit Exists}
The existence of a unit can be queried with the command \subcmdlink{unit}{exists}. 
This is the same as checking if a unit is in the result of \subcmdlink{unit}{names}, but is a more efficient implementation.
\begin{syntax}
\subcommand{unit}{exists} \$unitName
\end{syntax}
\begin{args}
\$unitName & Name of unit to query.
\end{args}
\subsection{Get Unit Info}
Information about a specific unit can be queried for any defined unit with the command \subcmdlink{unit}{info}.
Returns the ``unit info'', which is a Tcl dictionary with the following key/value pairs:\par
\begin{tabular}{lll}
\textbullet & value: & Conversion value to base units \\
\textbullet & dimension: & Dimension vector (force, length, time, temperature)
\end{tabular}\par
\begin{syntax}
\subcommand{unit}{info} \$unitName
\end{syntax}
\begin{args}
\$unitName & Name of unit to query.
\end{args}
\clearpage
\section{Importing Units as Variables}
Units can be imported as Tcl variables for use in Tcl expressions, using the command \subcmdlink{unit}{import}.
Variables created by the import command will be created the caller's scope (e.g. if called in a procedure, it will create local variables in that procedure), the variable names will match the unit names, and the variable values will represent the conversion from that unit to base units.

\begin{syntax}
\subcommand{unit}{import} \$arg1 \$arg2 ...
\end{syntax}
\begin{args}
\$arg1 \$arg2 ... & Units to import. -all for all units. 
\end{args}
\begin{example}{Importing selected units}
\begin{lstlisting}
unit system kip in sec
unit import ft ksi
puts $ft
puts $ksi
\end{lstlisting}
\tcblower
\begin{lstlisting}
12.0
1.0
\end{lstlisting}
\end{example}

\begin{example}{Importing all units}
\begin{lstlisting}
unit import -all
\end{lstlisting}
\end{example}
Note: As a safety measure, variable traces are added to unit variables to display a warning if overridden.
Also, be sure to import units \underline{after} defining the proper unit system. 
Redefining the unit system does not affect imported variables.
\clearpage
\section{Unit Strings}
The unit module uses a custom syntax for representing combinations of units, hereafter referred to as unit strings.
Unit strings may contain the following:
\begin{enumerate}[itemsep=0em]
\item Numbers (integer, decimal, and scientific)
\item Unit names (alphanumeric and underscore characters)
\item Whitespace
\item Parentheses \texttt{()}
\item Exponents \texttt{\caret}
\item Multiplication \texttt{*}
\item Division \texttt{/}
\end{enumerate}
For example, a unit string could be in\caret2, or kip*sec\caret2/in, or 1000*N/m, or gram/(cm*sec). Note that this format differs from Tcl expressions. Tcl expression math would be \$in**2, \$kip*\$sec**2/\$in, 1000*\$N/\$m, and \$gram/(\$cm*\$sec). The main difference is a lack of variable substitution and use of the caret symbol (\caret), instead of double asterisk (**), for exponentiation. The caret symbol has a different meaning than exponentiation in Tcl expressions.
\subsection{Getting Base Unit String for Unit Type}
The base-unit representation of a unit type can be queried with the command \subcmdlink{unit}{string}. 
\begin{syntax}
\subcommand{unit}{string} \$dimArg1 \$dimArg2 ...
\end{syntax}
\begin{args}
\$dimArg1 \$dimArg2 ... & Unit dimension arguments.
\end{args}
\begin{example}{Getting unit string for a unit type}
\begin{lstlisting}
unit system kip in sec
puts [unit string disp]
puts [unit string area]
\end{lstlisting}
\tcblower
\begin{lstlisting}
in
in^2
\end{lstlisting}
\end{example}
\clearpage
\subsection{Reduce Unit String to Base Unit String}
A unit string can be reduced to base-unit representation with the command \subcmdlink{unit}{reduce}.
\begin{syntax}
\subcommand{unit}{reduce} \$unitString
\end{syntax}
\begin{args}
\$unitString & Unit string to reduce.
\end{args}

\begin{example}{Reducing a unit string}
\begin{lstlisting}
unit system kip in sec
puts [unit reduce 20*kN/m^2]
\end{lstlisting}
\tcblower
\begin{lstlisting}
0.0029007547546041853*ksi
\end{lstlisting}
\end{example}

\subsection{Unit String Expressions}
The conversion value of a unit string to base units can be obtained using the \subcmdlink{unit}{expr} command.
Note that the expression syntax is limited in comparison to Tcl expressions. 
Only multiplication, division, exponents, and parentheses are allowed, with the caret symbol (\caret) representing exponentiation.
\begin{syntax}
\subcommand{unit}{expr} \$arg1 \$arg2 ...
\end{syntax}
\begin{args}
\$arg1 \$arg2 ... & Arguments that combine to form a unit string. Same style as Tcl \textit{expr} command.
\end{args}

\begin{example}{Unit expressions, two ways}
\begin{lstlisting}
unit system N m sec
puts [unit expr {10*cm}]
unit import cm
puts [expr {10*$cm}]
\end{lstlisting}
\tcblower
\begin{lstlisting}
0.1
0.1
\end{lstlisting}
\end{example}

\clearpage
\section{Unit Conversion}
The command \subcmdlink{unit}{convert} converts between compatible unit strings (equivalent base unit exponents) by computing a new conversion factor from the base unit conversions.
If the dimension vectors of the two unit strings are not compatible, it will throw an error.
\begin{syntax}
\subcommand{unit}{convert} <\$values> \$unitString1 \$unitString2
\end{syntax}
\begin{args}
\$values & Tcl list of values to convert. Default 1.0\\
\$unitString1 & Unit string to convert from \\
\$unitString2 & Unit string to convert to
\end{args}
\begin{example}{Unit conversions}
\begin{lstlisting}
# Independent of unit system
unit system N m sec
puts [unit convert 5 lbm kg]
unit system kip in sec
puts [unit convert 5 lbm kg]
# Works with unit strings
puts [unit convert 10 ft^2 in^2]
# Multiple conversion
puts [unit convert {10 20 30} lbf*in N*m]
\end{lstlisting}
\tcblower
\begin{lstlisting}
2.26796185
2.26796185
1440.0
1.129848290276167 2.259696580552334 3.3895448708285008
\end{lstlisting}
\end{example}
\clearpage

\section{Modifying the Unit System}
Although the unit module comes prepackaged with a set of built-in units, the unit set may be modified using the commands \subcmdlink{unit}{new}, \subcmdlink{unit}{combine}, \subcmdlink{unit}{remove} and \subcmdlink{unit}{wipe}. 
\subsection{Adding a New Unit}
The command \subcmdlink{unit}{new} creates a new unit with a base unit conversion value and base dimensions.
By default, base unit dimensions are assumed to be zero (constant).
Returns the unit info of the new unit.
\begin{syntax}
\subcommand{unit}{new} \$unitName \$value <\$dimArg1 \$dimArg2 ...>
\end{syntax}
\begin{args}
\$unitName & Name of unit to add \\
\$value & Conversion of unit to base units \\
\$dimArg1 \$dimArg2 ... & Unit dimension arguments. By default, unit is a constant. 
\end{args}
\subsection{Combining Existing Units}
The command \subcmdlink{unit}{combine} creates an aggregate unit using a unit string. With \subcmdlink{unit}{combine}, the base unit dimensions are automatically calculated.
Returns the unit info of the unit combination.
\begin{syntax}
\subcommand{unit}{combine} \$unitString <\$unitName>
\end{syntax}
\begin{args}
\$unitString & Unit string defining the unit. \\
\$unitName & Name of unit to add. By default, does not add a new unit but just returns the unit info.
\end{args}
\begin{example}{Two ways to create a new unit}
\begin{lstlisting}
unit system kip in sec
unit import kg m
puts [unit new rho [expr {$kg/$m**3}] mass 1 length -3]
puts [unit combine kg/m^3 rho]
\end{lstlisting}
\tcblower
\begin{lstlisting}
value 9.357254687402184e-11 dimension {1 -4 2 0}
value 9.357254687402184e-11 dimension {1 -4 2 0}
\end{lstlisting}
\end{example}
\clearpage
Note that both \subcmdlink{unit}{new} and \subcmdlink{unit}{combine} can be used to create the unit ``rho''. 
However, since ``kg'' and ``m'' were already created as units, \subcmdlink{unit}{combine} was a much simpler option, since base unit exponents did not need to be specified. 
Ideally, \subcmdlink{unit}{new} should only be used for base units; all other units can be derived using \subcmdlink{unit}{combine}.
\subsection{Removing Units}
Units can be removed with the command \subcmdlink{unit}{remove}. 
Note that this will not unset any variable associated with the unit if the unit was imported.
\begin{syntax}
\subcommand{unit}{remove} \$arg1 \$arg2 ...
\end{syntax}
\begin{args}
\$arg1 \$arg2 ... & Units to remove. -all for all units. 
\end{args}

\subsection{Wiping the Unit System}
All units and the unit system can be cleared with the command \subcmdlink{unit}{wipe}. 
Note that wiping the unit system will require that the command \subcmdlink{unit}{system} be called again before using other unit commands. 
\begin{syntax}
\subcommand{unit}{wipe}
\end{syntax}

\begin{example}{Defining unit system from scratch}
\begin{lstlisting}
unit wipe
unit system kip in sec
unit new ft 12.0 length
unit new ksi 1.0 stress
unit combine ksi/1000.0 psi
unit combine in/2.54 cm 
puts [unit names]
\end{lstlisting}
\tcblower
\begin{lstlisting}
kip in sec K ft ksi psi cm
\end{lstlisting}
\end{example}
\end{document}


